\documentclass[a4paper,12pt]{article}

\usepackage[pdftex]{graphicx}
\usepackage{amsmath,amsfonts,amsthm,amssymb}
\usepackage{fullpage}

\newtheorem{lemma}{Lemma}

\newcommand{\floor}[1]{\ensuremath{\left\lfloor #1 \right\rfloor}}

\author{April Camp 2017}
\title{Test 4 -- Solutions}
\date{}

\begin{document} \maketitle

\begin{enumerate}
	\item %China Western Invitational Mathematical Competition 2012
	\textit{Find the least positive integer $m$ such that for every prime number $p>3$ we have $105 \mid 9^{p^2} -29^p +m$.}
	
	Let $m$ be such an integer. Then since $105$ factorises into a product of primes as $3\times 5\times 7$, the given condition is equivalent to the condition that $3$, $5$ and $7$ each divide $9^{p^2} -29^p +m$ for every prime $p>3$.
	
	Now $0 \equiv_3 9^{p^2} -29^p +m \equiv_3 -(-1)^p+m = 1+m$ since $p$ is odd, so $m\equiv_3 -1$. Also, $0 \equiv_5 9^{p^2} -29^p +m \equiv_5 (-1)^{p^2} -(-1)^p +m = m$ since $p$ and $p^2$ are odd, so $m\equiv_5 0$. Finally, $0\equiv_7 9^{p^2} -29^p +m \equiv_7 2^{p^2} -1^p +m = 2^{p^2}-1+m$. Now since $p>3$ is prime, $3\nmid p$ and so $p^1 = 3k+1$ for some $k\in\mathbb{Z}$. Thus $2^{p^2} = 2\cdot 8^k \equiv_7 2$, and so $m\equiv_7 -1$.
	
	Now our three congruences for $m$ can be combined into one using the method of the Chinese Remainder Theorem. Alternatively, note that $3$, $5$ and $7$ all divide $m-20$, and so $m\equiv_105 20$. So the smallest positive integer value for $m$ is $m=20$.
	
	\item
	\textit{Determine all functions $f: \mathbb{R}\to\mathbb{R}$ such that for all $x,y\in\mathbb{R}$, $$f(xf(x)+f(xy)) = f(x^2) +yf(x).$$}
	We call this original equation King. Note that $f(x)=0$ for all $x\in\mathbb{R}$ is a solution to this equation since both sides of King are simply zero. So from now on we assume that $f$ has takes at least one nonzero value.
	
	Setting $x=0$ in King gives that $f(f(0)) = f(0)+yf(0)$. Since the left-hand side is constant the right-hand side is too, and so $f(0)=0$. Now let $e$ be a nonzero real number such that $f(e)\neq0$. Then substituting $e$ for $x$ and $y/e$ for $y$, King becomes $f(ef(e)+f(y)) = f(e^2)+y\times f(e)/e$. From this we see that since $f(e)/e\neq0$, $f$ is both injective and surjective. Then putting $y=0$ in King gives $f(xf(x)) = f(x^2) \iff xf(x)=x^2 \iff f(x)=x$ for all nonzero $x$ since $f$ is injective. Together with $f(0)=0$ this gives that $f(x)=x$ for all $x\in\mathbb{R}$.
	
	Substituting this into King we get that both sides become $x^2+xy$, so $f(x)=x$ is in fact a solution to King. Hence our only two solutions are $f(x)=0$ for all $x\in\mathbb{R}$ and $f(x)=x$ for all $x\in\mathbb{R}$.
	
	\item %Finnish MO 2014
	\textit{Determine the greatest positive integer $m$ such that each square of a $m\times m$ array can be painted either red or blue so that not all the squares at the intersection of any two rows and any two columns are the same colour.}
	
	For each row consider all pairs of squares from this row such that both squares are the same colour. From the condition of the problem it follows that no two rows can have any of these pairs in common. So, each pair can occur in at most one row. There are $\binom{m}{2} = \frac{m(m-1)}{2}$ pairs altogether and the squares in each pair are either both red or both blue. So there are $2 \cdot \frac{m(m-1)}{2} = m(m - 1)$ possible pairs.
	
	On the other hand we can estimate the lowest possible number of same-colour pairs in each row. Suppose that a row contains $k$ red and $m-k$ blue squares. Then the number of same-colour pairs is equal to $\binom{k}{2} + \binom{m-k}{2}$. This number can be bounded below by using the Quadratic-Arithmetic Means Inequality:
	\begin{align*} \binom{k}{2}+\binom{m-k}{2} &= \frac{k(k-1)}{2} +\frac{(m-k)(m-k-1)}{2} = \frac{k^2+(m-k)^2}{2} -\frac{m}{2} \\ &\geq \left(\frac{k+m-k}{2}\right)^2 -\frac{m}{2} = \frac{m(m-2)}{4}. \end{align*}
	Since the number of possible same-coloured pairs is at least as large as the total number of same-colour pairs in all the rows combined and we have bounded the number of same-colour pairs from below, we have $m(m-1) \geq m\cdot m(m-2)/4 \iff 0 \geq m^2-6m+4 \Rightarrow m \leq 5$ since $m$ is an integer.
	
	When $m = 5$ the lower bound from above has the value of $\frac{m(m-2)}{4} = \frac{15}{4} > 3$. This means that any colouring will create at least 4 same-colour pairs in each row. Since the number of possible pairs is $m(m-1) = 20$ and there are $5$ rows, each row has to contain exactly $4$ same-colour pairs. This can only happen when the row contains $3$ squares of one colour and $2$ squares of the other colour. Altogether we have to use all of the $10$ red pairs and all of the $10$ blue pairs. But the number of rows is odd. So, the number of squares of one colour will have to be greater than the number of squares of the other colour and the same will be true for the number of red pairs and the number of blue pairs. Hence, the $5 \times 5$ array cannot be coloured as desired.
	
	A $4\times 4$ array can be coloured as follows: for rows and columns numbered $1$ to $4$, the cells coloured blue are $(1,1),(1,2),(2,2),(2,3),(3,1),(3,4),(4,3),(4,4)$.
	
	\item %SL 2004
	\textit{In a cyclic quadrilateral $ABCD$, let $E$ be the intersection of $AD$ and $BC$ (so that $C$ is between $B$ and $E$), and $F$ be the intersection of $AC$ and $BD$. Let $M$ be the midpoint of side $CD$, and let $N\neq M$ be a point on the circumcircle of $\Delta ABM$ such that $\frac{AM}{MB}=\frac{AN}{NB}$. Show that $E$, $F$ and $N$ are collinear.}
	

\end{enumerate}

\end{document}
