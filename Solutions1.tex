\documentclass[a4paper,12pt, titlepage]{article}

\usepackage[pdftex]{graphicx}
\usepackage{amsmath,amsfonts,amsthm}

\newtheorem{lemma}{Lemma}

\newcommand{\grid}[4]{ %
    \begin{tabular}{|c|c|} %
        \hline %
        #1 & #2 \\
        \hline %
        #3 & #4 \\
        \hline %
    \end{tabular} %
}

\title{April Camp 2017 Test 1 -- Solutions}
\date{}

\begin{document} \maketitle

\begin{enumerate}
	\item 
	\textit{Determine all positive integers $k$ and $n$ satisfying the equation $$k^2 - 2016 = 3^n.$$}
	
    Note that if $n \geq 3$, then modulo $27$ the equation becomes $k^2 \equiv
    18 \pmod{27}$, which is a contradiction since $18$ is not a quadratic residue
    modulo $27$. Thus the only possible solutions correspond to $n=1$ and
    $n=2$. For $n=1$, the equation becomes $k^2 = 2019$, and for $n=2$, the
    equation is $k^2 = 2025 = 45^2$. We see that the only solution in
    positive integers is given by $n=2$ and $k=45$.
		
	\item 
	\textit{Let $ABC$ be an acute angled triangle. Let $H$ be the foot of the altitude from $C$ onto $AB$. Suppose that $|AH|=3|BH|$. Let $M$ and $N$ be the midpoints of the segments $AB$ and $AC$ respectively. Let $P$ be a point such that $|NP|=|NC|$ and $|CP|=|CB|$ and such that $B$ and $P$ lie on opposite sides of the line $AC$. Show that $\angle APM = \angle PBA$.}
	
	
	
	\item
	\textit{Consider a $4 \times 4$ grid of unit squares. How many ways are there to write a 0 or 1 in each $1 \times 1$ square so that the product of the two numbers written on every neighbouring pair of squares (sharing a common edge) is always 0?}
	
    Suppose that we have a collection of tiles, and in each tile is written
    either a $0$ or a $1$. We will call this configuration of tiles
    \emph{valid} if it satisifies the condition imposed in the problem:
    whenever two tiles share a border, the product of the numbers written
    in these two tiles is $0$. We wish to count the number of valid $4 \times
    4$ grids of tiles. We first prove two lemmas about the number of valid
    assignments of numbers to tiles in certain configurations.

    \begin{lemma} \label{lemma:strip}
        Let $a_n$ be the number of valid assignments of numbers to $n$ tiles
        placed in a line. (i.e. a $1 \times n$ strip of tiles.) Then $a_n =
        F_{n+2}$, where $F_n$ is the $n^\text{th}$ Fibonacci number.
    \end{lemma}
    \begin{proof}
        for $n=0$, there is $1$ valid assignment. (The ``empty
        assignment"), and for $n=1$, there are $2$ valid assignments. Thus
        $a_0=F_2$ and $a_1=F_3$. We now show that $a_{n} = a_{n-1} + a_{n-2}$,
        which then establishes the claim.

        Consider the number placed in the final tile in the strip. If it is a
        $0$, then we can place any numbers in the remaining tiles as long as
        the remaining $1 \times (n-1)$ strip is valid, and so there are
        $a_{n-1}$ valid configurations in this case. If the number in the final
        tile is a $1$,
        then this forces the penultimate tile to contain a $0$, and there
        are then no other restrictions other than that the remaining $1 \times
        (n-2)$ strip of tiles is valid. There are thus $a_{n-2}$ ways to validly
        assign the numbers in this case, and so we find that $a_n = a_{n-1} +
        a_{n-2}$.
    \end{proof}

    \begin{lemma} \label{lemma:ring}
        Let $b_n$ be the number of valid assignments of numbers to $n$ tiles
        fixed to a wall in a ring pattern. (i.e. the boarder of a
        rectangle.) Then
        $b_n = a_{n-1} + a_{n-3} = F_{n+1} + F_{n-1}$ for all $n \geq 4$.
    \end{lemma}
    \begin{proof}
        Consider any tile in the ring. If it contains a $0$, then the rest of
        the ring is a (bent) valid $1 \times (n-1)$ strip of tiles, and so we
        have that there are $a_{n-1}$ valid configurations in this case. On the
        other hand, if the chosen tile contains a $1$, then the two tiles
        which border it must both contain a $0$, and the remaining $(n-3)$
        tiles in the ring is now just any valid $1 \times (n-3)$ strip of
        tiles, and so there are $a_{n-3}$ valid configurations in this case.
     
        We see that $b_n = a_{n-1} + a_{n-3}$, and by Lemma \ref{lemma:strip},
        this is equal to $F_{n+1} + F_{n-1}$.
    \end{proof}

We now return to the problem. We consider three cases based on the number
of $0$'s contained in the central $2 \times 2$ subgrid of squares.

\begin{enumerate}

    \item[Case 1:] Four $0$'s: \grid{0}{0}{0}{0} \\
        
        Consider the ring of $12$ squares surrounding the central $2 \times
        2$ subgrid. The board as a whole is valid precisely when these $12$
        squares are, and so by Lemma \ref{lemma:ring}, we find that there
        are thus $F_{11} + F_{13} = 89 + 233 = 322$ valid grids in this case.

    \item[Case 2:] Three $0$'s: \grid{0}{0}{0}{1} \grid{0}{0}{1}{0}
        \grid{0}{1}{0}{0} \grid{1}{0}{0}{0} \\

        In this case, the squares on the border adjacent to the central square
        with a $1$ in must contain $0$'s. The square on the border which shares
        a corner with the square with a $1$ in can then be either a $1$ or a
        $0$, and in each of these cases, the remaining $9$ squares on the
        border can be any of the $a_9 = F_{11} = 89$ possible $1 \times 9$
        strips of valid tiles. In each of the $4$ configurations for the central
        $2 \times 2$ subgrid, we thus have $2 \times 89 = 178$ possible valid
        grids, giving us $178 \times 4 = 712$ valid configurations in this
        case.

    \item[Case 3:] Two $0$'s: \grid{0}{1}{1}{0} \grid{1}{0}{0}{1} \\

        The squares on the border adjecent to the $1$'s must be filled with
        $0$'s. The squares on the border sharing a corner with the $1$'s can be
        filled in any any way. For each of the $4$ possible ways of filling
        these squares, the remaining border squares form two independent $1
        \times 3$ strips of tiles, which can each be filled in $a_3 = F_5 =
        5$ ways, and so we see that the number of valid
        configurations in this case is $2 \times 4 \times 5^2 = 200$.

\end{enumerate}

Combining the above three cases, we find that the total number of valid $4
\times 4$ grids is given by $322 + 712 + 200 = 1234$.

	\item 
	\textit{Find all functions $f:\mathbb{R} \rightarrow \mathbb{R}$ satisfying $$f(xy-1)+f(x)f(y) = 2xy - 1$$ for all $x,y \in \mathbb{R}$.}
	
	
	
	\item
	\textit{Find all infinite sequences $a_1, a_2, a_3 \dots $ of positive integers such that
	\begin{itemize}
	\item [(a)] $a_{mn} = a_ma_n$ for all positive integers $m$ and $n$, and
	\item [(b)] there are infinitely many positive integers $n$ such that $$\{1, 2, \dots, n\} = \{a_1, a_2, \dots, a_n \}.$$
	\end{itemize}}
	
	
	
\end{enumerate}

\end{document}
