\documentclass[a4paper,12pt]{article}

\usepackage[pdftex]{graphicx}
\usepackage{amsmath,amsfonts}

\title{Test 2}
\date{Time: 4 hours}
\author{April Camp 2017}

\begin{document} \maketitle

\begin{enumerate}
	\item %Belarusian MO 2012
	For an integer $N>0$, $N$ boys, no two of them having the same height, are arranged in a circle. A boy in the given arrangement is said to be \emph{tall} if he is taller than both of his neighbours; a boy is said to be \emph{short} if he is shorter than both of his neighbours. Prove that the number of tall boys in the circle is equal to the number of short boys in the circle.
	
	\item %Belarusian MO 2012
	Nonzero real numbers $a,b,c,d$ satisfy the equations \[a+b+c+d = 0, \qquad \frac{1}{a}+\frac{1}{b}+\frac{1}{c}+\frac{1}{d}+\frac{1}{abcd} = 0.\]
	Find all possible values of the product $(ab-cd)(c+d)$.

	\item 

	\item 

	\item

\end{enumerate}

\medskip 

\hfill \emph{Each problem is worth 7 points.}
	
\end{document}

\item Find the largest integer $n$ satisfying the following conditions:
	\begin{enumerate}
		\item $n^2$ can be expressed as the difference of two cubes;
		\item $2n+79$ is a perfect square.
	\end{enumerate}

\item If $x$ and $y$ are rational numbers such that $x^5+y^5=2x^2y^2$, prove that $1-xy$ is the square of a rational number.
