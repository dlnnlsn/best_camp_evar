\documentclass[a4paper,12pt]{article}

\usepackage[pdftex]{graphicx}
\usepackage{amsmath,amsfonts,amsthm}
\usepackage{fullpage}

\newtheorem{lemma}{Lemma}

\newcommand{\floor}[1]{\ensuremath{\left\lfloor #1 \right\rfloor}}

\author{April Camp 2017}
\title{Test 3 -- Solutions}
\date{}

\begin{document} \maketitle

\begin{enumerate}
	\item 
	\textit{Given $5$ positive real numbers, show that there exist two of them,
	$a$ and $b$, for which \[0 \leq \frac{a}{1+a^2} - \frac{b}{1+b^2} \leq \frac{1}{8}.\]}
	
	
		
	\item 
	\textit{An acute-angled triangle $ABC$ is given. Points $D$, $E$, and $F$ lie
	on the sides $BC$, $CA$, and $AB$ respectively, and satisfy $\angle
	FDE = \angle BAC$. and $\angle DEF = \angle ABC$. Prove that the
	orthocenter of triangle $DEF$ coincides with the circumcenter of
	triangle $ABC$.}
	
	
	
	\item
	\textit{Prove that for every prime $p$, there exists an integer $x$ such that \[x^8 \equiv 16 \pmod p.\]}
	
	
	
	\item 
	\textit{Let $n > 3$ be an integer. John and Mary play the following game: First John labels the sides of a regular $n$-gon with the numbers $1, 2, \dotsc, n$ in whatever order he wants, using each number exactly once. Then Mary divides this $n$-gon into triangles by drawing $n − 3$ diagonals which do not intersect each other inside the n-gon. All these diagonals are labeled with number 1. Into each of the triangles the product of the numbers on its sides is written. Let $S$ be the sum of those $n − 2$ products.
	
Determine the value of $S$ if Mary wants the number $S$ to be as small as possible and John wants $S$ to be as large as possible and if they both make the best possible choices.}
	
	

\end{enumerate}

\end{document}
