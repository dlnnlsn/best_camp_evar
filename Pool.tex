\documentclass{article}

% Important Packages:
\usepackage{fullpage}
\usepackage{amsmath}    % need for subequations
\usepackage{amsfonts}
\usepackage{amsthm}
%\usepackage{graphicx}   % need for figures

% Useful macros 
\def\be{\begin{eqnarray}}	 	\def\ee{\end{eqnarray}}
\def\bea{\begin{eqnarray}}	 	\def\eea{\end{eqnarray}}
\def\bean{\begin{eqnarray*}}	\def\eean{\end{eqnarray*}}

\def\be{\begin{equation}}
\def\ee{\end{equation}}
\def\bea{\begin{eqnarray}}
\def\eea{\end{eqnarray}}
\def\bean{\begin{eqnarray*}}
\def\eean{\end{eqnarray*}}

\def\eg{e.\,g.}	% exempli gratia (for the sake of example)
\def\ie{i.\,e.}	% id est (that is)


\title{2017 Monthlies Pool}
 
\begin{document}	\maketitle

\begin{enumerate}
	\item Let $n$ be a positive integer. Determine the minimum number of lines that can be drawn on the plane so that they intersect in exactly $n$ distinct points. (Singapore 2016) 

	\item In triangle $ABC$ ($AB\ne AC$) the incircle with centre $I$ is tangent to side $BC$ at $D$. Let $M$ be the midpoint of side $BC$. Prove that the perpendiculars from points $M$ and $D$ to lines $AI$ and $MI$, respectively meet on the altitude in $\triangle ABC$ from $A$ or its extension. (Serbia 2016)
	
	\item Given a positive integer $n$, define $f(0,j)=f(i,0)=0$, $f(1,1)=n$ and
	\[ f(i,j) = \left\lfloor \frac{f(i-1,j)}{2}\right\rfloor + \left\lfloor \frac{f(i,j-1)}{2}\right\rfloor.\] 
	 for all integers $i,j\ge 0$, $(i,j)\ne (1,1)$. How many ordered pairs of positive integers $(i,j)$ are there for which $f(i,j)$ is an odd number? (Serbia 2016)
	
	\item A group of tourists get on 10 buses in the outgoing trip. The same group of tourists gets on 8 buses in the return trip. Assuming each bus carries at least 1 tourist, prove that there are at least 3 tourists who have taken a bus in the return trip that has more people than the bus he has taken in the outgoing trip. (Singapore 2016)
	
	\item Let $n\ge 3$ be an integer. Prove that it is impossible to find a set $\{a_1,\ldots, a_n\}$ of $n$ integers such that
	\[\{a_i+a_j \mid 1\le i\le j\le n\} \]
	leave distinct remainders modulo $n(n+1)/2$. (Singapore 2016) 
	
\end{enumerate}

\end{document}
	
	
	
	
	
\begin{enumerate}
	\item Let $x_1,\ldots, x_{2015}$ be non-negative real numbers such that $\ds\sum_{i=1}^{2015} x_i = 2014$. Determine the minimum  value of $\sum_{i=1}^{2015} x_i^i$. (DPRK 2015)
	
	\item Let $p$ be an odd prime and $a_1,\ldots, a_{p-2}$ positive integers satisfying
	\begin{itemize}
		\item for any $1\le k\le p-2$, neither $a_k$ nor $a_k^k-1$ is divisible by $p$ (they may be equal).
	\end{itemize}
	Prove that it is possible to choose a non-empty subset of these numbers whose product leaves a remainder of 2 when divided by $p$. (DPRK 2015)
	
	\item Anna creates $k$ disjoint pairs out of the numbers $n+1,n+2,\dots, n+2k$ arbitrarily. She tells Bob the product of the numbers in each of the pairs. Denote by $f(n)$ the maximal value of $k$ for which Bob can always figure out the pairs created by Anna out of these products. Prrove that there exist constants $c$ and $d$ independent of $n$ such that for every big enough value of $n$
	\[ c\sqrt{n} < f(n) < d\sqrt{n}.\]
	
\end{enumerate}

\end{document}
